\documentclass{beamer}

\usepackage{listings}

\title{Extensions and Combinators in Objective CAML}
\author{Dmitri Boulytchev}
\institute{St.Petersburg State University, Russia}
\date{LDTA'11 Tool Challenge, March 26-27, 2009, Saarbr\"ucken}

\lstdefinelanguage{Coolkit}{
keywords={let, rec, as, in, function, fun, match, with, val, ostap},
sensitive=true
}

\lstset{
basicstyle=\small,
identifierstyle=\ttfamily,
keywordstyle=\bfseries,
commentstyle=\scriptsize\rmfamily,
basewidth={0.5em,0.5em},
fontadjust=true,
escapechar=!,
language=Coolkit
}

\setbeamertemplate{footline}[frame number]
\setbeamertemplate{navigation symbols}{}

\begin{document}
\sloppy
\maketitle

\begin{frame}[fragile]
  \frametitle{Tool Challenge Solution Overview}

  \begin{itemize}
    \item T1-T3 $\times$ L1-L5;
    \item Parser combinators + syntax extension (\texttt{http://oops.math.spbu.ru/projects/ostap});
    \item Pretty-printer combinators;
    \item Extensible types (polymorphic variants).     
    \item Generic monadic transformers;
  \end{itemize}

  \begin{table}
\centering

\begin{tabular}{l|r}
\hline
 Files                   &   22 \\
 Lines of Code           & 2838 \\
 In Toplevels            &  973 \\
 In Language Components  & 1074 \\
 Commits                 &   14 \\
 Man-days                &    9 \\ 
  
\hline
\end{tabular}
\end{table}

\end{frame}

\begin{frame}[fragile]
  \frametitle{Ostap (parser combinator library)}
  \begin{itemize}
     \item Direct style monadic parser combinators;
     \item Syntax extension for OCaml;
     \item Higher-order parsers, streams with structural
subtyping. 
  \end{itemize}
  \begin{lstlisting}
   ostap (
    sample[a][b][c]: a "," b? DELIMITER c*  
   )

   val sample :
     (< getDELIMITER : ('a, 'b, < add : 'c -> 'c; .. > as 'c)
                      Combinators.result;
        look : string -> ('a, 'd, 'c) Combinators.result; 
        .. >
        as 'a) ->
     ('a, 'e, 'c) Combinators.parse ->
     ('a, 'f, 'c) Combinators.parse ->
     ('a, 'g, 'c) Combinators.parse ->
     ('a, 'e * 'd * 'f option * 'b * 'g list, 'c) 
        Combinators.result
  \end{lstlisting}
\end{frame}

\begin{frame}[fragile]
  \frametitle{Polymorphic Variants}
    \begin{lstlisting}
     val cattle : [< `Cow | `Sheep ] -> string 
     let cattle = function `Cow   -> "a cow" 
                         | `Sheep -> "a sheep" 

     let _ = 
       List.iter 
         (Printf.printf "%s\n") 
         (List.map cattle [`Cow; `Sheep; `Sheep; `Cow])
  \end{lstlisting}  
  
\end{frame}

\begin{frame}[fragile]
  \frametitle{Polymorphic Variants (continue)}
  \begin{lstlisting}
     val cattle : [< `Cow | `Sheep ] -> string 
     let cattle = function `Cow   -> "a cow" 
                         | `Sheep -> "a sheep" 

     let _ = 
       List.iter 
         (Printf.printf "%s\n") 
         (List.map cattle [`Cow; `Sheep; `Sheep; `Cow; 
                           `Cat; `Dog])
  \end{lstlisting}  


\end{frame}

\begin{frame}[fragile]
  \frametitle{Polymorphic Variants (continue)}
  \begin{lstlisting}
     val cattle : [< `Cow | `Sheep ] -> string 
     let cattle = function `Cow   -> "a cow" 
                         | `Sheep -> "a sheep" 
     val friend : [< `Cat | `Dog ] -> string
     let friend = function `Cat -> "a cat" 
                         | `Dog -> "a dog"

     let _ = 
       List.iter 
         (Printf.printf "%s\n") 
         (List.map (function 
                    | (`Cow | `Sheep) as x -> cattle x 
                    | (`Cat | `Dog  ) as x -> friend x)
            [`Cow; `Sheep; `Sheep; `Cow; `Cat; `Dog]
         )
  \end{lstlisting}  
\end{frame}

\begin{frame}[fragile]
  \frametitle{Polymorphic Variants (continue)}
  \begin{lstlisting}
     val cattle : ([> `Cow | `Sheep] as'a -> string) -> 
                  'a -> string 
     let cattle other = function `Cow   -> "a cow" 
                               | `Sheep -> "a sheep"
                               | x      -> other x 
     val friend : ([> `Cat | `Dog ] as 'a) -> string -> 
                  'a -> string
     let friend other = function `Cat -> "a cat" 
                               | `Dog -> "a dog"
                               | x    -> other x

     let rec all x = cattle (friend all) x

     let _ = 
       List.iter 
         (Printf.printf "%s\n") 
         (List.map all [`Cow; `Sheep; `Sheep; `Cow; 
                           `Cat; `Dog])
  \end{lstlisting}  
\end{frame}

\begin{frame}[fragile]
  \frametitle{Polymorphic Variants (continue)}
  \begin{lstlisting}
     val cattle : ([> `Cow | `Sheep] as'a -> string) -> 
                  'a -> string
     let cattle other = function `Cow   -> "a cow" 
                               | `Sheep -> "a sheep"
                               | x      -> other x 
     val friend : ([> `Cat | `Dog ] as 'a) -> string -> 
                  'a -> string
     let friend other = function `Cat -> "a cat" 
                               | `Dog -> "a dog"
                               | x    -> other x

     let rec fix f x = f (fix f) x
     let all y = fix (fun x -> cattle (friend x)) y 

     let _ = 
       List.iter 
         (Printf.printf "%s\n") 
         (List.map all [`Cow; `Sheep; `Sheep; `Cow; 
                           `Cat; `Dog])
  \end{lstlisting}  
\end{frame}

\begin{frame}[fragile]
  \frametitle{Drawbacks}
  
  \begin{itemize}
     \item Often types become ``too open'':
     \begin{lstlisting}
       val all : [> `Cat | `Cow | `Dog | `Sheep ] -> string
       let all y = fix (fun x -> cattle (friend x)) y 

       let _ = 
         List.iter 
           (Printf.printf "%s\n") 
           (List.map all [`Cow; `Sheep; `Sheep; `Cow; 
                          `Cat; `Dog; `Pig])
     \end{lstlisting}
     \item Type-error notification can be verbose.
  \end{itemize}
\end{frame}

\begin{frame}
  \begin{center}
  {\Large Thank you!}
  \end{center}
\end{frame}

\end{document}
